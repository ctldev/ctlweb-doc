% Hier wird das Deployment beschrieben.
% --------------------------------------------------------------------
\section{Deployment}
\subsection{Frontend}
Hier wird beschrieben, wie das Frontend des ctlweb-Projektes auf einem Ubuntu
12.04 LTS eingerichtet wird. Die Installation auf anderen Systemen ist mit
leicht anderen Befehlen ebenfalls möglich.

Benötigte Pakete: \begin{itemize}
  \item python-virtualenv
  \item python2.7-dev
\end{itemize}

zuerst wird eine neuer Benutzer angelegt. Dafür wird der Befehl
\begin{lstlisting}
sudo adduser ctlweb
\end{lstlisting}
ausgeführt. Für mehr Sicherheit sollte hier der Befehl
\begin{lstlisting}
sudo usermod -l ctlweb
\end{lstlisting}
ausgeführt werden, der das Passwort von ctlweb deaktiviert.

Als nächstes wird der Benutzer auf ctlweb gewechselt, um dort das frontend
einzurichten. Außerdem wird ein SSH-Schlüssel zur Kommunikation mit dem Backend
erstellt. 
\begin{lstlisting}
  sudo su - ctlweb
  ssh-keygen -t rsa -C ctl_backend_key
  git clone https://github.com/ctldev/ctlweb.git
  virtualenv -p python2 virtenv
  echo '. ~/virtenv/bin/activate' >> ~/.bashrc

  . ~/virtenv/bin/activate
  pip install django==1.4 django-registration gunicorn
\end{lstlisting}
Ebenfalls installiert werden muss die python-Bibliothek paramiko, allerdings
können hier bei der Installation über 
\begin{lstlisting}
pip install paramiko
\end{lstlisting}
Fehler auftreten. Sollte dies der Fall sein, muss das Paket über die
Paketverwaltung des Systems erfolgen, also zum Beispiel als root über den
Befehl
\begin{lstlisting}
apt-get install python-paramiko
\end{lstlisting}

Als nächstes werden die Einstellungen des Django-Projektes angepasst. Dazu
wird über
\begin{lstlisting}
cd ctlweb/src/frontend/settings
\end{lstlisting}
in den entsprechenden Ordner gewechselt. Anschließend wird die Datei
__init__.py über 
\begin{lstlisting}
nano __init__.py
\end{lstlisting}
angepasst, so dass folgendes in der Datei steht
\begin{lstlisting}
from .deploy import *
\end{lstlisting}
anschließend wird die deploy.py erzeugt, der Einfachheit halber wird dazu die
dev.py kopiert und anschließend bearbeitet. Über den Befehl 
\begin{lstlisting}
cp dev.py deploy.py
\end{lstlisting}
erfolgt das kopieren, dann über 
\begin{lstlisting}
nano deploy.py
\end{lstlisting}
bearbeiten, dass folgendes in der Datei steht. Werte in '<>' sollten dabei ohne
<> mit den benötigten Werten ersetzt werden.
\begin{lstlisting}
#vim: set fileencoding=utf-8
"""Development settings and globals."""

from default import *
from os.path import join, normpath

# Local time zone for this installation. Choices can be found here:
# http://en.wikipedia.org/wiki/List_of_tz_zones_by_name
# although not all choices may be available on all operating systems.
# On Unix systems, a value of None will cause Django to use the same 
# timezone as the operating system.
# If running in a Windows environment this must be set to the same as your
# system time zone.
TIME_ZONE = '<Europe/Berlin>'

# Language code for this installation. All choices can be found here:
# http://www.i18nguy.com/unicode/language-identifiers.html
LANGUAGE_CODE = '<de-de>'

SITE_DOMAIN = '<http://ctlweb.example.net>'

DEBUG = False
TEMPLATE_DEBUG = DEBUG

DATABASES = {
        'default': {
            'ENGINE': 'django.db.backends.sqlite3',
            'NAME': normpath(join(SITE_ROOT, 'app', 'ctlweb', 'ctlweb.db')),
            'USER': '',
            'PASSWORD': '',
            'HOST': '',
            'PORT': '',
        }
        #prefer mysql?
        #'default': {
            #'ENGINE': 'django.db.backends.mysql',
            #'NAME': '<ctlweb>', #database name
            #'USER': '<ctlweb>', #database user
            #'PASSWORD': '<password>',
            #'HOST': '', #empty for localhost
            #'PORT': '', #empty for default
        #}
    }

ACCOUNT_ACTIVATION_DAYS = 7

PAGINATION_PAGE_RANGE_INTERFACES = 8
PAGINATION_PAGE_RANGE_COMPONENTS = 5
PAGINATION_BUTTON_RANGE = 3

SSH_KEY_FILE = /home/ctlweb/.ssh/ctl_backend_key
SSH_KEY_PASSWORD = ''

# A sample logging configuration. The only tangible logging
# performed by this configuration is to send an email to
# the site admins on every HTTP 500 error.
# See http://docs.djangoproject.com/en/dev/topics/logging for
# more details on how to customize your logging configuration.
LOGGING = {
    'version': 1,
    'disable_existing_loggers': False,
    'handlers': {
        'mail_admins': {
            'level': 'ERROR',
            'class': 'django.utils.log.AdminEmailHandler'
        }
    },
    'loggers': {
        'django.request': {
            'handlers': ['mail_admins'],
            'level': 'ERROR',
            'propagate': True,
        },
    }
}

DEFAULT_FROM_EMAIL = '<ctlweb@example.net>'
EMAIL_HOST = '<smtp.example.net>'
EMAIL_HOST_USER = '<ctlweb@example.net>'
EMAIL_HOST_PASSWORD = '<password>'
EMAIL_USE_TLS = True
EMAIL_PORT = 587
\end{lstlisting}

anschließend wird die Datenbank und das Impressum erstellt.
\begin{lstlisting}
cd ..
python manage.py syncdb
python manage.py create_impressum
\end{lstlisting}

Nun wird ein Upstart-Skript erstellt, dass es ermöglicht, die Webseite als
Service laufen zu lassen und dadurch einen automatischen Start nach dem
Herunterfahren des Servers wieder zu gewährleisten. Das Skript wird über 
\begin{lstlisting}
sudo nano /etc/init/ctlweb-frontend
\end{lstlisting}
bearbeitet, folgender Inhalt sollte danach eingetragen sein
\begin{lstlisting}
description "CtlWeb-Frontend starten und stoppen"
version "1.0"
author "Theodor van Nahl, Philipp Offensand"
start on runlevel [2345]
stop on runlevel [016]

setuid ctlweb
chdir /home/ctlweb/
respawn

script
. /home/ctlweb/virtenv/bin/activate
exec python manage.py run_gunicorn 127.0.0.1:4444
end script
\end{lstlisting}
Nun kann das frontend über 
\begin{lstlisting}
sudo service ctlweb-frontend start
\end{lstlisting}
gestartet werden.
Weiterhin muss das ausgeführte Django von einem echten Webserver weitergereicht
werden. Erklärt wird dies hier über den leichtgewichtigen Webserver lighttpd. 
Dazu wird über den Befehl
\begin{lstlisting}
sudo nano /etc/lighttpd/conf-available/20-ctlweb.conf
\end{lstlisting}
die Datei bearbeitet, in der folgendes stehen könnte
\begin{lstlisting}
\$HTTP["host"] =~"ctlweb.example.net" {
  server.document-root = "/home/ctlweb/ctlweb/src/frontend/app/ctlweb/static"

  \$HTTP["url"] !~ "/static/" {
    proxy.balance = "hash"
    proxy.server = ("" => ("CtlWeb" =>
                           ("host" => "127.0.0.1", "port" => 4444 )
    ))
  }

  alias.url = (
    "/static/" => "/home/ctlweb/ctlweb/src/frontend/app/ctlweb/static/",
  )

  url.rewrite-once = (
    "^/favicon.ico\$" => "/static\$1",
    "^/static.*\$" => "\$1",
  )
}
\end{lstlisting}

Auf der Webseite sollten nun in der Admin-Section die Cluster hinzugefügt werden,
auf denen sich die darzustellenden Komponenten befinden. Die Komponenten des
Clusters lassen sich über den Befehl
\begin{lstlisting}
python manage.py request\_modules
\end{lstlisting}
abfragen. Regelmäßige Abfragen sind so auch zum Beispiel über einen Cron-Job
oder ähnliche Mechanismen zu realisieren.

Nun sollte das Frontend voll einsatzfähig sein.
\subsection{Backend}
Hier werden nun die Schritte beschrieben, die nötig sind, um das Backend des
ctl-web Systems voll einsatzfähig zu machen.

Zunächst ist es erforderlich Python 3 zu installieren, sowie die Virtual
Environment Funktionalität hinzuzufügen.
\begin{lstlisting}
  sudo apt-get install python3 git python-virtualenv
\end{lstlisting}
Dann wird ein neuer User für das ctl-System angelegt
\begin{lstlisting}
  sudo adduser ctl
\end{lstlisting}
sowie das Passwort deaktiviert, was zu einer Erhöhung der Sicherheit führt.
\begin{lstlisting}
  sudo usermod -l ctl
\end{lstlisting}
Es wird der User zu ctl gewechselt und das ctlweb-backend wird installiert.
Dazu gehört auch das Erzeugen neuer ssh-keys und die Anpassung einiger
Systempfade.
\begin{lstlisting}
  sudo su - ctl
  ssh-keygen -t rsa -C ctl_backend_key
  git clone https://github.com/ctldev/ctlweb.git
  virtualenv -p python3 virtenv
  echo 'export PYTHONPATH=${HOME}/ctlweb/lib/backend/:${PYTHONPATH}' >> \
    ~/virtenv/bin/activate
  echo 'export PATH=${HOME}/ctlweb/bin:${PATH}' >> ~/virtenv/bin/activate
  echo '. ~/virtenv/bin/activate' >> ~/.bashrc

  . ~/virtenv/bin/activate
  pip install requests
\end{lstlisting}

Nachdem nun das Backend erfolgreich installiert wurde, folgt nun die
Einrichtung desselbigen. Dazu wird die Konfigurationsdatei angepasst.
\begin{lstlisting}
  cp ~/ctlweb/etc/ctlweb.conf ~/.ctlweb.conf
  vi ~/.ctlweb.conf
\end{lstlisting}

Schließlich muss nur noch der Ordner für den Manifest Store sowie die Datei für
die Datenbank angelegt werden.
\begin{lstlisting}
  mkdir <Manifest_store>
  touch <Database>
\end{lstlisting}

Nun ist das Backend erfolreich installiert und einsatzbereit!


