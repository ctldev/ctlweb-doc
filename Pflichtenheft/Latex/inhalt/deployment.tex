% Hier wird das Deployment beschrieben.
% --------------------------------------------------------------------
\chapter{Deployment}
Es gibt viele Pakate die für den funktionierenden Server nötig sind. Im folgenden wird beschrieben, wie (und ggf. wo) diese zu installieren sind. 


Zuerst muss das Projekt geforkt werden. Dazu im gewünschten Ordner den Befehl 
\begin{lstlisting}
git clone https://github.com/ctldev/ctlweb.git 
\end{lstlisting}
ausführen. Im Ordner des
Projektes befindet sich im Subordner 'src/frontend' die manage.py von Django.
hier sollte der Befehl 
\begin{lstlisting}
python manage.py create\_impressum
\end{lstlisting}
 ausgeführt werden, um ein personalisiertes Impressum zu generieren. Alternativ muss ein eigenes 
in dem Ordner 'src/frontend/template' mit dem namen
'personal\_impressum.html' generiert werden. Ab jetzt kann die Webseite gestartet
werden, indem man im Ordner 'src/frontend' den Befehl
\begin{lstlisting}
python manage.py runserver 
\end{lstlisting}
eingibt. Es fehlen aber noch einige Pakete um diese vollständig zu nutzen.
Für das Datenbanksystem muss das SQLite-Paket installiert werden
\begin{lstlisting}
sudo apt-get install sqlite3
\end{lstlisting}
Als nächsten Schritt installiert man 
Auf der Webseite sollten nun in der Admin-Section die Cluster hinzugefügt werden,
auf denen sich die darzustellenden Komponenten befinden. Die Komponenten des
Clusters lassen sich über den Befehl '
\begin{lstlisting}
python manage.py request\_modules
\end{lstlisting}
abfragen. Regelmäßige Abfragen sind so auch zum Beispiel über einen Cron-Job
oder Ähnliches zu realisieren.
\newpage
Der nächste Schritt ist Django 1.4 zu installieren, indem man den Befehl
\begin{lstlisting}
sudo pip install django==1.4
\end{lstlisting}
ausführen lässt. 
Nun ist es wichtig pip zu installieren. Folgender Befehl erledigt das:
\begin{lstlisting}
sudo get-apt python-pip
\end{lstlisting}
Weiterhin brauch der Server die Django-registration, die wie folgt hinzugefügt wird:
\begin{lstlisting}
sudo pip install django-registration
\end{lstlisting}
Jetzt sind nur noch 2 Sachen zu erledigen. Einmal ist das Paramiko-Paket zu installieren 
\begin{lstlisting}
sudo install paramiko
\end{lstlisting}
und zuletzt muss noch der Sync-Befehl durchgeführt werden um die Datenbank vorzubereiten.
\begin{lstlisting}
python manage.py syncdb
\end{lstlisting}
Nun sollte die Webseite voll einsatzfähig sein.
