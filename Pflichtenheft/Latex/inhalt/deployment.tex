% Hier wird das Deployment beschrieben.
% --------------------------------------------------------------------

\chapter{Deployment}
\section{Frontend}
Folgende Pakete werden für die funktionsfähigkeit der Webseite benötigt:
  \begin{itemize}
	\item paramiko
	\item django-registration
  \end{itemize}


Zuerst muss das Projekt geforkt werden. Dazu im gewünschten Ordner den Befehl '
git clone https://github.com/ctldev/ctlweb.git' ausführen. Im Ordner des
Projektes befindet sich im Subordner 'src/frontend' die manage.py von Django.
hier sollte der Befehl 'python manage.py create\_impressum' ausgeführt werden,
um ein personalisiertes Impressum zu generieren. Alternativ muss ein eigenes 
in dem Ordner 'src/frontend/template' mit dem namen
'personal\_impressum.html' generiert werden. Nun kann die Webseite gestartet
werden.
Auf der Webseite sollten nun in der Admin-Section die Cluster hinzugefügt werden,
auf denen sich die darzustellenden Komponenten befinden. Die Komponenten des
Clusters lassen sich über den Befehl 'python manage.py request\_modules'
abfragen. Regelmäßige Abfragen sind so auch zum Beispiel über einen Cron-Job
oder Ähnliches zu realisieren.
\section{Backend}
Folgendes Paket wird für die Funktionsfähigkeit des Programms benötigt:
\begin{itemize}
  \item requests (python3)
\end{itemize}
