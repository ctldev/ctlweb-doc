% Kapitel 4
%-------------------------------------------------------------------------------

\chapter{Produktfunktionen}

\section{Frontend Funktionen}
\subsection{F100 (Anmeldung)}
\label{F:Anmeldung}
\begin{description}
  \item[Ziel]Ein Benutzer meldet sich erfolgreich am System an.
  \item[Vorbedingung]Der Benutzer ist bereits auf dem System registriert und besitzt ein Passwort.
  \item[Akteure]Unangemeldeter Benutzer
   \item[Beschreibung]\hfill
    \begin{enumerate}
      \item Der Benutzer wählt den Link \emph{Login} auf der Webseite aus.
	  \item Der Benutzer kommt in ein Login-Fenster, in dem er Benutzer und
		Passwort eingeben kann und bestätigt dies mit einem Button.
	  \item Die Daten werden vom System überprüft und der Benutzer informiert
		ob der Login erfolgreich war.
	  \item Außerdem wird dem Benutzer irgendwo auf der Seite sein Login-Status
		angezeigt.
    \end{enumerate}
\end{description}

\subsection{F101 (Suche)}
\label{F:Suche}
\begin{description}
  \item[Ziel]Eine bestimmte Komponente wird gefunden
  \item[Vorbedingung]Der Benutzer ist angemeldet
  \item[Akteure] Angemeldeter Benutzer
   \item[Beschreibung]\hfill
    \begin{enumerate}
      \item Der Benutzer wählt den Link \emph{Suche} auf der Webseite aus.
	  \item Der Benutzer hat die Wahl zwischen zwei Suchverfahren.
	  \item Das Erste ist eine einzelne Eingabezeile in der mit logischen
		Begriffen gesucht werde kann.
	  \item Das Zweite besteht aus mehreren optionalen Suchfeldern, die
		relevante Eigenschaften der Komponenten repräsentieren. Hier kann der
		Benutzer gezielt nach Eigenschaften der Komponenten suchen.
	  \item Die Ergebnisse werden in einer übersichtlichen Tabelle mit den
		wichtigsten Eigenschaften angezeigt.
    \end{enumerate}
\end{description}

\subsection{F102 (Detailinformation der Komponenten)}
\label{F:Details}
\begin{description}
  \item[Ziel]Einem Benutzer werden Detailinformationen einer Komponente
	angezeigt.
  \item[Vorbedingung]Der angemeldete Benutzer besitzt z.B. durch Verwenden der Suche den
	Link zur Detailseite der Komponente.
  \item[Akteure]Angemeldeter Benutzer
   \item[Beschreibung]\hfill
    \begin{enumerate}
      \item Der Benutzer öffnet den Link zur Detailseite einer Komponente.
	  \item Dem Benutzer werden in tabellenartiger Form alle
		Detailinformationen angezeigt.
	  \item Der Benutzer erhält die Möglichkeit eine Verbindung zu der
		Komponente aufzubauen.
    \end{enumerate}
\end{description}

\subsection{F103 (Eintragung von Komponenten)}
\label{F:Eintragen}
\begin{description}
  \item[Ziel]Eine neue verfügbare Komponente wird eingetragen.
  \item[Vorbedingung]Der Benutzer ist als User angemeldet und hat alle
	relevanten Daten einer neuen Komponente, die bereitgestellt werden soll.
  \item[Akteure]Administrator
   \item[Beschreibung]\hfill
    \begin{enumerate}
      \item Der Benutzer wählt den Link \emph{Komponentenadministration} auf der Webseite aus.
	  \item Der Benutzer kommt in ein Fenster in dem er alle notwendigen und
		gewünschte zusätzliche Informationen einträgt.
	  \item Der Benutzer bestätigt die Informationen und das System trägt die
		neue Komponente in seine Datenbank ein.
	  \item Der Benutzer bekommt Informationen ob die Eintragung erfolgreich
		verlaufen ist.
    \end{enumerate}
\end{description}


\subsection{F104 (Bearbeiten von Komponenten)}
\label{F:Bearbeiten}
\begin{description}
  \item[Ziel]Eine Komponente wird bearbeitet oder gelöscht.
  \item[Vorbedingung]Der Benutzer ist als Administrator auf der Detailseite
	einer Komponente oder als User auf der Detailseite einer selbst eingetragenen
	Komponente.
  \item[Akteure]Administrator, User
   \item[Beschreibung]\hfill
    \begin{enumerate}
      \item Der Benutzer hat einen zusätzlichen Button zur Bearbeitung auf der
		Detailseite der Komponente.
	  \item Mit dem Button kommt der Benutzer auf eine Seite, auf der er alle
		relevanten Informationen ändern kann
	  \item Auf der Seite befindet sich ein Button zum Löschen der Komponente.
	  \item Der Benutzer bestätigt seine Änderungen mit einem bestimmten
		Button.
    \end{enumerate}
\end{description}

\begin{itemize}
	\item Login /Logout
	\item Suche nach Componten von Benutzern
	\item Manifest Push
	\item Manifest pull
	\item Manifest upload
	\item Snippet bereitstellung
	\item 
\end{itemize}<++>
