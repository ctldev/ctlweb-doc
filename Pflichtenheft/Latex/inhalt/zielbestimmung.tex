% Kapitel 1
% Die Unterkapitel können auch in separaten Dateien stehen,
% die dann mit dem \include-Befehl eingebunden werden.
%-------------------------------------------------------------------------------

\chapter{Zielbestimmung}

%Hier Einleitungstext einfügen, dabei die Formatierungen selber erstellen


Das Ziel unseres Projektes ist es, ein System zu erstellen, welches es unserer Benutzergruppe erlaubt, eigene Komponenten bereitzustellen bzw. Komponenten anderer im eigenen Projekt zu verwenden, sowie die Ausführung dieser Komponenten auf Computerclustern. Für die Anzeige der Komponenten wird ein Webinterface bereitgestellt, das Ausführen der Komponenten übernimmt ein zweites paralleles System.
Die Benutzergruppe umschließt vorerst die TU-Braunschweig, evtl. auch andere Universitäten sowie Institutionen. 
 

\section{Musskriterien}

%\begin{itemize}
%\item  Es muss eine Webfrontend geben um die Compontents einzusehen
%\item  Webfrontend muss initialen Quellcode Snippets bereit stellen.
%\item  Stabilität des Systems
%\item  Push für Manifest ins Frontend
%\item  Pull von Cluster der Manifestdateien
%\item  Upload von Manifest ins Frontend
%\end{itemize}

\begin{itemize}
\item Webinterface für die "Discovery", d.h. das Webinterface besitzt eine gut funktionierende Suche, welche dem Nutzer als Ergebnis die gewünschten Komponenten anzeigt
\item Die Detailseite der Komponenten enthält erste wichtige Informationen wie Name, Autor, Version etc. auf, weiterhin wird eine Quellcode
Snippet bereitgestellt
\item Rollen: 
\begin{description}
\item{Gast}kann nach Komponenten suchen, kann deren Details ansehen
\item{registrierter Nutzer} kann Komponenten bei sich einfügen, kann "Programme" mit diesen Fremdkomponenten starten
\item{Administrator} kann das Webinterface anpassen, d.h. Komponenten freischalten/löschen, Nutzer freischalten/löschen etc.
\end{description}
\item ein zweites System, welches die Verbindungen der Komponenten zu den Clustern bereitstellt
\end{itemize}

\paragraph{Benutzerdaten}

\begin{description}
	\item[Administrator] Alles
	\item[User]
		\begin{itemize}
			\item Kann Components ausführen
			\item Neue C hinzufügen
			\item Auslesen von C
			\item Login auf Webfrontend
		\end{itemize}
	\item[Gast] 
		\begin{itemize}
			\item Manifest lesen
			\item Kein Snippet-Code lesen
			\item Componten lesen
		\end{itemize}
\end{description}

\section{Wunschkriterien}

\begin{itemize}
\item  Absicherung des Zugriffes
\item  Anbindung von diesen Systemen untereinander (Social Programming)
\item  LDAP Anbindung
\item  Multilinguale Unterstützung
\end{itemize}

\section{Abgrenzungskriterien}

\begin{itemize}
\item  Social Programming wird nicht implemtiert, soll nur evtl. vorbereitet werden.
\item  Das System ist nicht für sehr große
\item  Es wird \emph{nur} English als Systemsprache zur Verfügung
\end{itemize}
