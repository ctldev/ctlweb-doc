% Kapitel 1
% Die Unterkapitel können auch in separaten Dateien stehen,
% die dann mit dem \include-Befehl eingebunden werden.
%-------------------------------------------------------------------------------

\chapter{Zielbestimmung}

%Hier Einleitungstext einfügen, dabei die Formatierungen selber erstellen
\begin{itemize}
\item  System für Programmkomponenten
\item  Bereitstellung von CTL (Component Template Library)
\item  Zugriff auf CTL
\item  Für intra-universitären Austausch
\end{itemize}

\section{Musskriterien}

\begin{itemize}
\item  Es muss eine Webfrontend geben um die Compontents einzusehen
\item  Webfrontend muss initialen Quellcode Snippets bereit stellen.
\item  Stabilität des Systems
\item  Push für Manifest ins Frontend
\item  Pull von Cluster der Manifestdateien
\item  Upload von Manifest ins Frontend
\end{itemize}

\paragraph{Benutzerdaten}

\begin{description}
	\item[Administrator] Alles
	\item[User]
		\begin{itemize}
			\item Kann Components ausführen
			\item Neue C hinzufügen
			\item Auslesen von C
			\item Login auf Webfrontend
		\end{itemize}
	\item[Gast] 
		\begin{itemize}
			\item Manifest lesen
			\item Kein Snippet-Code lesen
			\item Componten lesen
		\end{itemize}
\end{description}

\section{Wunschkriterien}

\begin{itemize}
\item  Absicherung des Zugriffes
\item  Anbindung von diesen Systemen untereinander (Social Programming)
\item  LDAP Anbindung
\item  Multilinguale Unterstützung
\end{itemize}

\section{Abgrenzungskriterien}

\begin{itemize}
\item  Social Programming wird nicht implemtiert, soll nur evtl. vorbereitet werden.
\item  Das System ist nicht für sehr große
\item  Es wird \emph{nur} English als Systemsprache zur Verfügung
\end{itemize}
